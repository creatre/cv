%        File: cv.tex
%     Created: Fri Sep 21 10:00 AM 2012 C
% Last Change: Fri Sep 21 10:00 AM 2012 C
%
\documentclass[a4paper]{article}
\begin{document}
Mikael Vatau
Kornvägen 33
232 38  Arlöv
Sweden


I am 38 years and specialized in Embedded Systems Programming and Linux. I work mainly with C and have an acquired taste for Linux/Unix. My preferred tools of the trade is vim, zsh, and strong coffee. I am tired of cold weather, quite evenings and a comfortable life. I am kind, tolerant and likable. I prefer good food, and happy people. 

Mar 2005 - today
Senior Consultant, Netville AB

Jan 2000 - Sep 2004
Research Engineer, Lund University, Robotics


Education

Malmö University	Practical English in Cross-cultural Communication	2003 - 2003			
Lund Institute of Technology	Electro Engineer (B.Sc.E.E.)	1995 - 2000			

Courses
Linux for Embedded Systems Advanced	Sony Ericsson Mobile Communications AB	Enea	2009		
Linux for Embedded Systems	            Ericsson AB	External consultant company	     	2008		
GPRS Security				Ericsson AB	Ericsson Internal			2006	
IRIX System administration	SGI		SGI							2000



Date: 	Oct 2010 – today
ST-Ericsson
Description: 	Low level Audio in the audio/hw team on Android platform. Involved in low level audio  The work involves debugging and solving issues related to audio in the kernel and drivers. Also responsible for the LTP test cases for the alsa driver. Worked mostly with audiocodecs and low power audio.

Date: 	Jun 2010 - Sep 2010
Unnamed customer
Android application to help visually impaired people in making phone calls. 

 
Mar 2009 - Apr 2010
Sony Ericsson Mobile Communications AB		
Developer of embedded test software for testing hardware components in a mobile phone and worked mainly with GPS, WLAN and touch display. As part of the Embedded Test Software team helped bring up drivers and write the test-software used at the production site. The ETS software was used  when manufacturing developer prototypes used internally and in the high volume manufacturing.
 
Mar 2005 - May 2008	: 
Ericsson AB	
C, Lint, Visual Studio, Eclipse
Worked with configuration the PCLint tool for the platform (OSE) and compiler. PCLint is a static code analysis tool that flag suspicious constructs in the code and was used by all the developers before checking in a change. Supported developer with advice or by changing the compiler.

Ericsson AB	Mar 2005 - May 2007	
Worked with integration of component subsystems in a Clearcase environment at the Open Platform Architecture (OPA) department. Work also include regression testing, release and test documentation, and trouble shooting, (The trouble shooting was mostly to find out who was the responsible for a regression and included lauterbach debugger, dump analysis and looking at logs). 

Jan 2000 - Sep 2004
University of Lund in Sweden, Robotics institution		
Irix, Linux, Windows
Network administration for the robotics department. This work was done in parallel with the other tasks mentioned below.
 

January 2000 – September 2004
University of Lund in Sweden, Robotics institution

Worked in several projects, and in several roles, among these:

Apr 2003 - Sep 2004	
PLC, CANopen, Isagraf Designer
Developed software for a laboratory robot used by AstraZeneca for protein testing. The robot would take a protein sample, deploy it, and then clean the picker. Software was done in PLC using the Isagraph Designer.
 
Jul 2003 - Feb 2004
PHP, MySQL, Javascript, HTML
Developed a new webpage for the Robotics department which included news section, document repository. PHP, MySQL and Javascript was used. 
 
Jan 2000 - Jun 2003
C, Protel, P\&E Microsystem, CAN, Frescale MCU, Unix
Redesigned  controller cards that where used to control DC and Step motors. This involved designing    the micro controller card with a 32 bit CPU from Freescale in the Protel Cad software and verify / troubleshoot the hardware. The CPU used a CAN bus to communicate with other controller cards and the PC.  Also developed the software running on the embedded system. The micro-controller was used in several different projects e.g. a robotics arm on a wheelchair.
\end{document}


